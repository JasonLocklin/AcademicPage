
%%%%%%%%%%%%%%%%%%%%%%%%%%%% Document Setup %%%%%%%%%%%%%%%%%%%%%%%%%%%%
\documentclass[10pt]{article}
% This is a helpful package that puts math inside length specifications
\usepackage{calc}

% Layout: Puts the section titles on left side of page
\reversemarginpar

%% Letter sized paper
\usepackage[paper=letterpaper,
            %includefoot, % Uncomment to put page number above margin
            marginparwidth=1.2in,     % Length of section titles
            marginparsep=.05in,       % Space between titles and text
            margin=1in,               % 1 inch margins
            includemp]{geometry}

%% Use these lines for A4-sized paper
%\usepackage[paper=a4paper,
%            %includefoot, % Uncomment to put page number above margin
%            marginparwidth=30.5mm,    % Length of section titles
%            marginparsep=1.5mm,       % Space between titles and text
%            margin=25mm,              % 25mm margins
%            includemp]{geometry}

%% More layout: Get rid of indenting throughout entire document
\setlength{\parindent}{0in}

%% This gives us fun enumeration environments. compactenum will be nice.
\usepackage{paralist}

%% Reference the last page in the page number
%
% NOTE: comment the +LP line and uncomment the -LP line to have page
%       numbers without the ``of ##'' last page reference)
%
% NOTE: uncomment the \pagestyle{empty} line to get rid of all page
%       numbers (make sure includefoot is commented out above)
%
\usepackage{fancyhdr,lastpage}
\pagestyle{fancy}
%\pagestyle{empty}      % Uncomment this to get rid of page numbers
\fancyhf{}\renewcommand{\headrulewidth}{0pt}
\fancyfootoffset{\marginparsep+\marginparwidth}
\newlength{\footpageshift}
\setlength{\footpageshift}
          {0.5\textwidth+0.5\marginparsep+0.5\marginparwidth-2in}
\lfoot{\hspace{\footpageshift}%
       \parbox{4in}{\, \hfill %
                    \arabic{page} of \protect\pageref*{LastPage} % +LP
%                    \arabic{page}                               % -LP
                    \hfill \,}}

% Finally, give us PDF bookmarks
\usepackage{color,hyperref}
\definecolor{darkblue}{rgb}{0.0,0.0,0.3}
\hypersetup{colorlinks,breaklinks,
            linkcolor=darkblue,urlcolor=darkblue,
            anchorcolor=darkblue,citecolor=darkblue}

%PDF specific settings:
\usepackage[pdftex,
        pdftitle={curriculum vitae},
        pdfauthor={Jason Locklin},
        pdfproducer={pdfLaTeX}]{}

%Definitions (ie., \item[]) should start on a new, indented line
\usepackage{enumitem}
\setdescription{labelsep=\textwidth}


%%%%%%%%%%%%%%%%%%%%%%%% End Document Setup %%%%%%%%%%%%%%%%%%%%%%%%%%%%


%%%%%%%%%%%%%%%%%%%%%%%%%%% Helper Commands %%%%%%%%%%%%%%%%%%%%%%%%%%%%

% The title (name) with a horizontal rule under it
%
% Usage: \makeheading{name}
%
% Place at top of document. It should be the first thing.
\newcommand{\makeheading}[1]%
        {\hspace*{-\marginparsep minus \marginparwidth}%
         \begin{minipage}[t]{\textwidth+\marginparwidth+\marginparsep}%
                {\large \bfseries #1}\\[ -0.15\baselineskip]%
                 \rule{\columnwidth}{1pt}%
         \end{minipage}}


\renewcommand{\section}[2]%
        {\pagebreak[2]\vspace{1.3\baselineskip}%
         \phantomsection\addcontentsline{toc}{section}{#1}%
         \hspace{0in}%
         \marginpar{
         \raggedright \scshape #1}#2}

% An itemize-style list with lots of space between items
\newenvironment{outerlist}[1][\enskip\textbullet]%
        {\begin{enumerate}[#1]}{\end{enumerate}%
         \vspace{-.6\baselineskip}}

% An itemize-style list with little space between items
\newenvironment{innerlist}[1][\enskip\textbullet]%
        {\begin{compactenum}[#1]}{\end{compactenum}}

% To add some paragraph space between lines.
% This also tells LaTeX to preferably break a page on one of these gaps
% if there is a needed pagebreak nearby.
\newcommand{\blankline}{\quad\pagebreak[2]}

%%%%%%%%%%%%%%%%%%%%%%%% End Helper Commands %%%%%%%%%%%%%%%%%%%%%%%%%%%


%%%%%%%%%%%%%%%%%%%%%%%%% Begin CV Document %%%%%%%%%%%%%%%%%%%%%%%%%%%%

\begin{document}
\makeheading{Jason Locklin, BSc, MA}

\section{Contact Information}
\newlength{\rcollength}\setlength{\rcollength}{2.5in} %right column width

\begin{tabular}[t]{@{}p{\textwidth-\rcollength}p{\rcollength}}
\href{http://psychology.uwaterloo.ca/}
     {Department of Psychology} & \\
\href{http://www.uwaterloo.ca/}{University of Waterloo}
                           & \textit{Voice:} (519) 888-4567 x36662\\
100 University Ave.            & \textit{Fax:} (519) 746-8631 \\
Waterloo, Ontario           & \textit{E-mail:} \href{mailto:jalockli@uwaterloo.ca}{JaLockli@UWaterloo.ca}\\
N2L 3G1    & \textit{WWW:}
\href{http://artsweb.uwaterloo.ca/~jalockli}{artsweb.uwaterloo.ca/\~{}jalockli}\\
\end{tabular}

%\section{Research Interests}  %comment out for teaching
%
%Visuomotor Control, Perception, Brain Injury, Object Tracking.

\section{Education}
\href{http://www.uwaterloo.ca/}{{University of Waterloo}}

\begin{outerlist}

\item[] \href{http://psychology.uwaterloo.ca}
             {\textbf{Ph.D., Cognitive Neuroscience}}, (Expected completion August 2013)
        
        \begin{innerlist}
        \item Thesis Topic: Perception and Action Biases under Saccadic and Prism Adaptation.
        \item Adviser:
              \href{http://psychology.uwaterloo.ca/people/faculty/jdancker/index.html}
                   {Professor James Danckert}
        \item Area of Study: The neuroanatomy and psychophysics of human action and perception.
        \end{innerlist}

\item[] \href{http://psychology.uwaterloo.ca}
             {\textbf{M.A., Behavioural and Cognitive Neuroscience}}, August 2009
        
        \begin{innerlist}
        \item Thesis Title: Development of a Measure of Visuomotor Control for Assessing the Long-term Effects of Concussion
        \item Adviser:
              \href{http://psychology.uwaterloo.ca/people/faculty/jdancker/index.html}
                   {Professor James Danckert}
        \item Area of Study: Fine Motor Control, Concussion Research.
        \end{innerlist}

\item[] \href{http://science.uwaterloo.ca}
        {\textbf{B.Sc. (Honours)}},
             Major: Psychology, Minor: Biology, June 2007
        \begin{innerlist}
        \item Course load including strong mix of Psychology and Natural Science: Biology, Physics, Chemistry, Biochemistry, Organic Chemistry, and Calculus.
        \item Completed laboratory courses associated with every Natural Science course taken.
        \item Achieved a 95\% in Advanced Data Analysis.
        \item Average GPA over final 4 terms 84\%.
        \end{innerlist}

\end{outerlist}

\section{Awards}
%
\href{http://www.uwaterloo.ca}{Government of Ontario}
\begin{outerlist}
\item[]\begin{tabular}[t]{@{}p{\textwidth-\rcollength}p{\rcollength}}
Ontario Graduate Scholarship & 2010--2011\\
Aiming for the Top Tuition Scholarship & 2002--2003\\
\end{tabular}
\end{outerlist}
\blankline %\vspace{5 mm}

\href{http://www.uwaterloo.ca}{University of Waterloo}
\begin{outerlist}
\item[]\begin{tabular}[t]{@{}p{\textwidth-\rcollength}p{\rcollength}}
President's Graduate Scholarship & 2010--2011\\
UW/Faculty of Arts Graduate Scholarship & 2010\\
Psychology Memorial Fund Scholarship & 2010\\
Arts Graduate Enhancement Scholarship & 2009\\
University of Waterloo Merit Scholarship & 2009\\
MERIT/Faculty of Arts Graduate Scholarship & 2008--2009 \\
Arts Grad Enhancement Scholarship & 2007--2008\\
Dean of Science Honours List (non-monetary) & 2004 \\
Dean of Science Honours List (non-monetary) & 2005 \\
\end{tabular}
\end{outerlist}


% Include the content from markdown files with pandoc
% See makefile
$body$


\section{Teaching}
\href{http://psychology.uwaterloo.ca}{Department of Psychology}, University of Waterloo:

\begin{outerlist}

\item[] \textbf{Advanced Data Analysis}: \textit{Teaching Assistant \& Lab Instructor}%
        \hfill \textbf{Fall 2007}
\begin{innerlist}
\item Instructor: Jonathan Fugelsang
\item Develop and lead regular 1 hour tutorials instructing 30 students to utilize the statistical software package SPSS in analyzing real world experimental and observational data.
\item Graded weekly assignments and tests.
%\item Course Description: \\ \texttt{Aimed at developing an understanding of the use and interpretation of statistics in complex research designs. Emphasis on analysis of variance and multiple comparison techniques to interpret the results of multi-factor experiments. The importance of power in factorial designs will be discussed. The course includes a computer component that ties the use of a statistical package to the topics discussed in lectures.}
\end{innerlist}

\item[] \textbf{Basic Data Analysis}: \textit{Teaching Assistant \& Lab Instructor}
        \hfill \textbf{Winter 2008}
\begin{innerlist}
\item Instructor: Derek Koehler
\item Instruct a weekly tutorial for 30 students, consisting of a 30 minute review lecture of the week's topic, and 30 minutes of practical instruction on solving data analytic problems.
\item Develop weekly tutorial lesson plans in cooperation with other TAs.
\item Create marking keys for tests and grade students.
%\item Course Description:\\ {\texttt{An introduction to the logic and methods of descriptive and inferential statistics with emphasis on application in Psychology. Topics covered include measures of central tendency and variability, distributions, the normal distribution, z-scores, hypothesis testing, probability, chi-square tests, t-tests, power, and correlation and regression.}
\end{innerlist}

\item[] \textbf{Cognitive Processes}: \textit{Teaching Assistant}%
        \hfill \textbf{Fall 2008}
\begin{innerlist}
\item Instructor: Jonathan Fugelsang
\item Provide extra instruction during office hours, grade term papers, give feedback to students.
\end{innerlist}

\item[] \textbf{Physiological Psychology}: \textit{Teaching Assistant}%
        \hfill \textbf{Winter 2009 and Fall 2009}
\begin{innerlist}
\item Instructor: Erin Skinner.
\item Provide weekly office hour extra instruction to students.
\end{innerlist}


\item[] \textbf{Res. in Human Cognitive Neuroscience}: \textit{Teaching Assistant}%
        \hfill \textbf{Winter 2010}
\begin{innerlist}
\item Instructor: Mike Dixon.
\item Provide assistance and feedback to students developing a research paper.
\end{innerlist}

\item[] \textbf{Physiological Psychology}: \textit{Teaching Assistant}%
        \hfill \textbf{Winter 2011}
\begin{innerlist}
\item Instructor: James Danckert
\item Provide weekly office hour extra instruction to students.
\end{innerlist}
\end{outerlist}


\section{Research}
\href{http://psychology.uwaterloo.ca}{Department of Psychology}, University of Waterloo:

\begin{outerlist}
\item[] \textit{Research Assistant}%
        \hfill \textbf{May 2007 to Current}
\begin{innerlist}
\item Supervisor: Dr. James Danckert
\item Develop a motor-accuracy task for the measurement of concussion symptoms.
\item Test neurological patients using a variety of neuropsychological tests and procedures, including Prism Adaptation.
\item Develop and test a gaze-contingent task using real-time eye-tracking equipment.
\end{innerlist}


\item[] \textit{Research Assistant}%
        \hfill \textbf{Jan. 2007 to Apr. 2007}
\begin{innerlist}
\item Supervisor: Dr. Jon Fugelsang
\item Develop web-based decision making experiments and collect data.
\end{innerlist}


\item[] \textit{Laboratory Coordinator}%
        \hfill \textbf{Sep. 2005 to Aug. 2006}
\begin{innerlist}
\item Supervisor: Dr. Scott McCabe
\item Train research assistants, coordinate lab events, oversee several experiments.
\end{innerlist}


\item[] \textit{Research Assistant}%
        \hfill \textbf{Jun. 2005 to Dec. 2006}
\begin{innerlist}
\item Supervisor: Dr. James Danckert
\item Develop a computer-based task for participants to pursue moving targets on a touch-screen
computer.
\end{innerlist}

\end{outerlist}


\section{Other Relevant Experience}
\begin{outerlist}
 \item[] \textit{U.W. Undergraduate Psychology Society}: President%
        \hfill \textbf{Sep. 2005 to Apr. 2007}
\end{outerlist}



\section{Mathematical Expertise}
\begin{outerlist}
\item[] \textit{Basic Statistics and Data Analysis}
\begin{innerlist}
 \item Hypothesis testing via means comparisons and correlations, including techniques for the prevention of elevated experiment-wise error.
 \item Data reduction and simplification using measures of central tendency, variance, and periodicity.
 \item Data visualization, including experience with the problem of communicating high-dimensional data on paper/screen.
 \item Experiment power and effect size calculations.
 \item Experimental design optimization.
\end{innerlist}

\item[] \textit{Advanced Statistics}
\begin{innerlist}
 \item Analysis of Variance and Covariance, as well as Logistic Regression (Generalized Linear Models).
 \item Multiple Regression, including model comparisons and variable coding for non-typical data sets.
 \item Bayesian hypothesis testing.
\end{innerlist}
\end{outerlist}

\section{Technical Skills}
\begin{outerlist}
\item[] \textit{Statistical / Data Analytical Software}
\begin{innerlist}
 \item \href{http://www.spss.com/}{SPSS}, a statistical analysis package for the social sciences, 
 \item \href{http://www.r-project.org/}{R}, a powerful statistics programming language and data visualization package.
 \item \href{http://www.scipy.org/}{Scipy/Pylab}, a Python library for scientific computation and data visualization.
\end{innerlist}

\item[] \textit{Laboratory Equipment and Software}
\begin{innerlist}
 \item \href{www.sr-research.com}{EyeLink II} host control (For controlling eye-tracking equipment).
 \item \href{http://www.psychopy.org/}{Psychopy}, a Python library for building psychopysics and cognitive psychology experiments.
 \item Authored a library to interface the EyeLink host with Psychopy.
 \item Trained in lesion overlay analysis using a variety of software packages (Analysis of fMRI brain imaging data)
 \item AcqKnowlege software in combination with BIOPAC laboratory equipment for physiological measurement (Electromyography (EMG) and Galvanic Skin Response (GSR))
 \item \href{http://www.pstnet.com/products/e-prime/}{E-Prime} for data collection in Psychology Research.
\end{innerlist}

\item[] \textit{Programming and Scripting Languages}
\begin{innerlist}
 \item Day-to-day familiarity with several scripting languages including Python, UNIX shell scripting (BASH), and BASIC (\href{http://www.pstnet.com/products/e-prime/}{E-Basic}).
 \item Experience using a variety of programming languages (C, Pascal, and JAVA).
 \item Experience using PHP with HTML and CSS in the development of web-based research experiments.
\end{innerlist}


\item[] \textit{Typesetting and Productivity Software}
\begin{innerlist}
 \item Comfortable writing with \TeX{}, \LaTeX{}, and B\textsc{ib}\TeX{} for technical and scientific documents, as well as common office suites such as Microsoft Office and OpenOffice.
\end{innerlist}


\end{outerlist}








\end{document}

%%%%%%%%%%%%%%%%%%%%%%%%%% End CV Document %%%%%%%%%%%%%%%%%%%%%%%%%%%%%
